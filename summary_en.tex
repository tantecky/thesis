\section*{Summary}
The following thesis is dealing with simulation of suspension in a mechanically stirred tank by using computational fluid dynamics. The main goal was to determine the height of suspension cloud and asses the impact of the model for the drag coefficient on the distribution of solid phase. The studied system consisted of a flat bottom cylindrical tank with an inner diameter $T=\SI{0.29}{\meter}$. The solid phase was formed of PVC particles and its concentration was 5, 10 and \volfrac{15}. Tap water and polyvinylpyrrolidone were used as working liquid. The height of filling was chosen to $H=T$ and agitation was provided by pitched six-blade turbine (the pitch angle \SI{45}{\degree}). The simulations were done by commercial software \flu{} 12.1.4, which contained several drag coefficient models implemented in the form of user-defined functions. Eulerian-Eulerian and Eulerian-Granular approaches together with standard \keps{} turbulence modele were used for a description of the multiphase flow. The obtained results corresponded well with experimental measurements. Especially good agreement was found for the simulations when the volume fraction of the solid phase was equal to \proc{5} or \proc{10}.
