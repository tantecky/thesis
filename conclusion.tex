\chapter{Závěr}
Tato práce byla zaměřena na použití metody CFD k~simulaci suspendace v~mechanicky míchané nádobě. Hlavním cílem bylo simulačně stanovit výšku vznosu pevné fáze pro několik odlišných konfigurací promíchávaného systému. V~rámci této studie bylo navíc porovnáno několik modelů pro výpočet koeficientu odporu v~turbulentní oblasti proudění, který byly implementovány do řešiče \flu{} v~podobě uživatelem definovaných funkcí.

Standardní \keps{} turbulentní model se ukázal jako dostačující k~popisu proudění v~nádobě osazené čtyřmi radiálními narážkami. Při porovnání jednotlivých korelací pro koeficient odporu vyplynulo, že daný model ovlivňuje distribuci pevné fáze zvláště při vyšších objemových zlomcích dané fáze. S~rostoucím stupněm homogenizace směsi se rozdíly mezi jednotlivými korelacemi vytrácely. Model navržený Brucatem pro Tay\-lo\-rův–Couetteovův tok však vykazoval výrazně odlišné chování, a proto ho nelze doporučit pro simulaci suspendace v~mechanicky míchané nádobě. Zbývající modely projevovaly velmi podobné chování se zvyšující se dobou simulace. V~případech, kdy koncentrace pevné fáze dosahovala \volproc{5} nebo \volproc{10} pevné fáze, byla výška suspezního mraku stanovená pomocí techniky CFD v~dobré shodě s~experimentálně získanými výsledky. Nicméně znatelnější rozdíly mezi simulací a experimentem byly pozorovány, když objemový zlomek kuliček z~PVC dosahoval \proc{15}. Po srovnání více\-fá\-zo\-vých modelů se ukázalo, že přístup Eulerian-Eulerian poskytuje při nižší koncentraci zrnité fáze téměř totožné výsledky jako model Eulerian-Granular. Simulačně stanové doby homogenizace byly téměř vždy vyšší v~porovnání s~experimentálním měřením, což je způsobeno podhodnocení turbulentních veličin u~\keps{} turbulentního modelu. Dále se simulačně podařilo potvrdit trend nárůstu doby homogenizace vlivem vyšší dynamické viskozity použité vsádky.

Další výzkum v~oblasti simulace suspendace v~míchaných nádobách se může zaměřit na úpravu či návrh vlastního modelu pro koeficient odporu za účelem dosažení lepší predikce výšky suspenzního mraku při vyšších koncentracích pevné fáze. V~úvahu také připadá rozšíření následující práce na trojfázový systém kapalina-plyn-pevná fáze v~ae\-ro\-va\-ných mí\-cha\-cích ná\-do\-bách.
