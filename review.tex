\chapter{Literární rešerše}
Suspendace pevné fáze v~mechanicky míchaných nádobách patří v~posledních desetiletích ke značně studovanému jevu. Za průkopníka v~této oblasti je označován \citet{zwi58}, jenž vyvinul empirickou korelaci pro výpočet frekvence otáčení míchadla při které již dochází ke vznosu částic. Při této frekvenci žádná částice nezůstává na dně nádoby déle než jednu či dvě sekundy. Také další autoři intenzivně experimentálně studovali mechanizmus suspendace v~promíchávaných nádobách \citep{nie68,bal78,arm98}. 

V~posledních třech dekádách mnoho výzkumníků obrátilo svoji pozornost k~počítačové dynamice tekutin jako k~prostředku ke zkoumání procesů probíhajících ve vícefázových systémech. Od té doby technika CFD nabývá na vzrůstající oblibě jako základního nástroje ke studiu tokového pole a distribuce pevných částic v~míchaných nádobách. Rané CFD simulace byly pouze zaměřeny na určení rychlostního pole v~jednofázových systémech \citep{kre91}. Experimentální metody jako LDV byly používány k~určení nutných okrajových podmínek pro modelování pohybu míchadla. Hlavní nevýhodou takto získaných simulací je především jejich úzký vztah se zkoumaným systémem. Další vývoj na poli CFD se zaměřil na vývoj simulačních metod, které by kompletně odstranili tuto závislost na experimentálních datech. Tyto snahy vyústily k~tvorbě komplexních technik jako je metoda vícenásobných souřadnicových soustav (MRF) nebo metoda klouzající sítě (SM), které jsou schopny popsat rotaci míchadla bez dalších dodatečných dat.  

Zatímco jednofázové toky v~míchaných nádobách byly podrobně zkoumány, jak experimentálně, tak výpočetně. Mnohem méně prací bylo publikováno na téma vícefázového proudění v~míchaných systémech. Následující kapitola shrnuje několik významných prací, který byly provedeny za účelem studie suspendace pevné fáze v~mechanicky promíchávaných nádobách.
  
\section{Suspendace pevné fáze v~míchaných nádobách}
\citet{lju01} simulovali dvoufázové proudění v~nádobě se šestilopatkovým míchadlem se šikmo skloněnými lopatkami. Zkoumaný systém byl tvořen válcovou nádobou s~plochým dnem a čtyřmi radiálními narážkami. Vnitřní průměr této nádoby činil $T=\SI{0.297}{\meter}$ a jako vsádka byla použita voda. Zrnitou fázi tvořily skleněné částice o~průměru od \SI{150}{\micro\meter} do \SI{450}{\micro\meter}. Pro simulaci vícefázového proudění autoři zvolil techniku Eulerian-Eulerian spolu se standardním \keps{} turbulentním modelem. Výpočetní doména se přibližně skládala ze \num{52000} buněk a díky symetrii systému, pouze čtvrtina nádoby byla simulována. K~popisu mezifázové odporové síly výzkumníci využil čtyři odlišné korelace pro koeficient odporu. Tyto modely navrhli autoři: \citet{schi32}, \citet{ish79}, \citet{ihme72} a \citet{bru98}. Získaná simulační data byla porovnána s~experimentálními údaji naměřenými pomocí fázové Dopplerové anemometrie. V~získaných výsledcích se mezifázový odpor projevil jako dominantní silou působící na částice pevné fáze. Na druhou stranu zdánlivá setrvačná síla (angl. virtual mass force) a Saffmanova vztlaková síla měly po celou dobu zanedbatelný vliv na výsledky simulace. Člen mezifázové turbulentní disperze především ovlivňoval řešení v~blízkosti osy nádoby, avšak ve zbytku systému byl jeho působení zanedbatelné. Závěrem autoři konstatovali, že všechny použité modely pro koeficient odporu poskytovali velmi podobné výsledky.

Dalším způsobem jak modelovat suspendaci pevné fáze je využití vícefázového modelu Eulerian-Granular. Zmíněnou techniku využili autoři \citet{oshi02} ke studiu distribuce pevné fáze v~míchané nádobě, kde koncentrace částic se pohybovala v~rozmezí od \volproc{0.5} do \volproc{50}. Navíc v~této práci stanovili výšku vznosu pevné fáze, dobu homogenizace a tzv. kvalitu suspendace, což je směrodatná odchylka koncentrace pevné fáze. Zkoumaný systém se skládal z~válcové nádoby do které bylo ponořeno čtyřlopatkové míchadlo se šikmo skloněnými lopatkami. Pro snížení výpočetní náročnosti byla simulační doména uvažována jako dvojrozměrná spolu s~osovou symetrií. Pro popis turbulence byl využit standardní \keps{} model a viskozita zrnité fáze byla vypočtena pomocí korelace navržené \citet{syam93}. Získané výsledky byly v~dobré shodě s~experimentálními daty, jenž byly nalezeny v~literatuře. Avšak nesrovnalosti byly pozorovány mezi korelací pro minimální frekvencí otáčení míchadla potřebnou ke dosažení vznosu a touto frekvencí určenou ze simulačních dat. Autoři tento rozpor připisovali faktu, že v~práci byly použiti moderní vysoce účinná axiální míchadla spolu s~poměrné malou světlou výškou ode dna nádoby během experimentů. Nicméně směrodatná odchylka objemového zlomku pevné fáze se ukázala jako užitečné měřítko míry kvality suspendace.        

\citet{derk03} provedl sérii simulací kapalina a pevná fáze pomocí matematického modelu Eulerian-Lagrangian. Navíc autor využil techniku velkých vírů (LES) k~popisu suspendace v~mechanicky míchané nádobě pomocí Rushtonovy turbíny. Míchací nádoba měla standardní konfiguraci, jenž byla tvořena válcovou nádrží s~plochým dnem o~průměru $T=\SI{0.297}{\meter}$. Jako pracovní médium byla použita voda do které byly ponořeny částice pevné fáze o~velikosti \SI{300}{\micro\meter}. Za zmínku stojí fakt, že ve všech provedených simulacích byla koncentrace pevné fáze relativně malá (do \volproc{3.6}). Získané výsledky kvalitativně korespondovali s~experimentálním měřením, které provedl \citet{miche03} pro podobný systém kapalina-pevná fáze. Absence Saffmanovy vztlakové síly a zdánlivé setrvačné síla neměla žádný významný dopad na výsledné koncentrační profily. Autor pozoroval, že pouze pokud jsou zahrnuty do simulace interakce mezi jednotlivými částicemi, tak distribuce pevné fáze v~nádobě má realistický charakter. 

Výška suspenze je jednou z~významných charakteristik promíchávaných systémů. Autoři \citet{mic04} provedli sérii experimentálních měření výšky vznosu pevné fáze pro různé rychlosti otáčení míchadla v~průhledné nádobě. Průměr nádoby činil $T=\SI{0.19}{\meter}$ a hladina kapaliny byla vyšší než obvykle, aby se usnadnilo pozorování oblasti ve které se nenacházejí žádné částice. Kuličky oxidu křemičitého o~velikosti v~rozmezí \num{212}--\SI{250}{\micro\meter} byly použity jako pevná fáze a jejich množství se lišilo od \volproc{0.45} do \volproc{14.4}. Kromě experimentů byly provedeny simulace výše popsaného systému za účelem zjištění schopnosti predikce tohoto jevu pomocí CFD. V~práci byl použit vícefázový model Eulerian-Eulerian spolu s~technikou klouzající sítě pro popis pohybu Rushtonovy turbíny. Díky symetrické povaze úlohy pouze jedna polovina nádoby byla simulována a výsledná výpočetní doménu tvořilo \num{52992} šestistěnných buněk. Vysoce turbulentní proudění bylo modelováno pomocí homogenního \keps{} přístupu, při kterém obě fáze sdílejí stejné hodnoty turbulentní kinetické energie a rychlosti disipace turbulentní energie. V~této práci byla použita dobře známá korelace pro odporový koeficient, kterou navrhli \citet{schi32}. Provedené experimenty ukázaly téměř lineární závislost mezi výškou suspenzního mraku a frekvencí otáčení míchadla. Ze získaných výsledků vyplynulo, že model Eulerian-Eulerian je schopen kvalitativně popsat výšku vznosu zrnité fáze. Na závěr autoři doporučili, pro zlepšení shody s~experimentálními výsledky, zohlednění turbulence v~modelu pro koeficient odporu a zahrnutí interakcí mezi jednotlivými částicemi.

Většina provedených CFD simulací systémů kapalina-pevná fáze se skládá z~nádoby opatřené narážkami a jedním rychloběžným míchadlem. Avšak někteří autoři se ve svých výzkumech neomezili pouze na toto uspořádání. Jako příkladem může posloužit studie, jenž provedli \citet{mon04}. V~této práci autoři zkoumali suspendaci pevné fáze ve válcové nádobě, která obsahovala čtyři míchadla, a to, buď Rushtonovy turbíny, nebo míchadla typu Lightnin A310. Navíc při některých experimentech nádrž neobsahovala radiální narážky, takže docházelo k~tvorbě vysoce vířivého proudění. K~vlastní simulaci byl využit vícefázový model Eulerian-Granular spolu s~korelací pro koeficient odporu navržený \citet{pin01}, který zohledňuje turbulentní proudění. Pro srovnání byly výpočty provedeny i s~klasický vztah pro koeficient odporu, jenž navrhli autoři \citet{schi32}. V~tomto případě však docházelo k~předpovědi nižší hodnoty koeficientu odporu, což mělo za následek kratší dobu usazování částic než bylo pozorováno při experimentálních měřeních. Turbulentní model \keps{} se ukázal jako dostatečně přesný pro případ nádrže s~narážkami. Naopak pro systém bez narážek tento model vykazoval nerealistické chování rychlostní pole a místo něj musel být využit výpočetně náročnější Reynoldsův napěťový model (angl. Reynold's stress model -- RMS). Z~výsledků vyplynulo, že zvláštní pozornost musí být věnována výběru korelace pro koeficient odporu, jenž se jeví jako kritický parametr pro správnost predikce distribuce pevné fáze.

\citet{ochi08} se studovali vliv několika odporových koeficientů na rozložení rychlosti a koncentraci pevné fáze v~mechanicky míchané nádobě. Experimentální aparatura se skládala z~válcové nádrže o~vnitřním průměru $T=\SI{0.378}{\meter}$ do které bylo ponořeno míchadlo Mixtec HA735 o~světlé výšce $C=\num{0.15}T$. Jako pevná fáze byly použity částice niklu o~průměrech \SI{230}{\micro\meter}, \SI{400}{\micro\meter} a \SI{750}{\micro\meter}, přičemž jejich objemový zlomek v~nádrži se pohyboval v~rozmezí od \volproc{1} do \volproc{20}. K~matematickému popisu systému kapalina-pevná fáze byl použit přístup Eulerian-Granular společně s~modelem \keps{} pro turbulenci. Celkem tři odlišné korelace pro koeficient odporu byly použity v~této práci \citep{schi32, bru98, gid94}, přičemž poslední jmenovaný se často používá při simulaci fluidace. Autoři poukázali na fakt, že hodnota rychlosti disipace kineticé energie vypočtena pomocí CFD je nižší než tato hodnota určená experimentálně z~příkonu míchadla. Toto snížení má za následek nižší korekci na turbulentní režim proudění v~Brucatově vztahu, a proto se tento model blíží v~řadě případů korelaci, kterou navrhli \citet{schi32}. Ze získaných výsledků vyplynulo, že v~průměru nejlepší shody s~experimentálními údaji byla dosaženo při použití Brucatovy korelace zohledňující volnou turbulenci. Model dle Gidaspowa však překonával v~kvalitě predikce Brucatovu korelaci při vyšších koncentracích pevné fáze (\volproc{20}). Tento fakt je způsoben zohledněním objemového zlomku pevné fáze ve výše zmíněném modelu.

Další studii systému kapalina-pevné fáze v~mechanicky míchané nádobě provedli autoři \citet{kas08}. Ve své práci zkoumali především kvalitu suspendace, výšku suspenzního mraku, dobu homogenizace a distribuci pevné fáze v~nádobě s~Rushtonovou turbínou a čtyřmi radiálními narážkami. Na závěr získané simulační výsledky byly porovnány s~experimentálními údaji získanými výzkumníky \citet{yama08}. Zkoumaný systém se skládal z~válcové nádoby o~vnitřním průměru $T=\SI{0.3}{\meter}$, jenž byla naplněna vodou do výšku $H=T$. Zrnitou fází tvořily skleněné částice o~středním průměru \SI{264}{\micro\meter}, hustotě \SI{2470}{\kilogram\per\cubic\meter} a objemovém zlomku \volproc{10}. Vzhledem k~symetrii systému pouze polovina nádoby tvořila výpočetní doménu, která celkem obsahovala \num{298000} šestistěnných buněk. K~vlastní simulaci vícefázového proudění byl využit přístup Eulerian-Eulerian spolu se standardním homogenním \keps{} modelem pro turbulenci. Pro popis mezifázové odporové síly byla využita korelace navržená autory \citet{kho06}, jenž vznikla modifikací Brucatova vztahu. Rotující míchadlo bylo modelováno pomocí techniky MRF a jeho frekvence otáčení byla měněna v~intervalu od \SI{2}{\per\second} do \SI{40}{\per\second}. Ze získaných výsledků vyplynulo, že použitý výpočetní model je schopen správně kvantitativně popsat závislost mezi frekvencí otáčení míchadla a kvalitou suspendace (resp. distribuci pevné fáze). Stanovená doba homogenizace se nejprve zvyšovala s~rostoucí rychlostí otáčení míchadla. Avšak po dosažení svého maxima, začala prudce klesat a při stavu homogenní suspenze nabývala téměř konstantní hodnoty. Autoři určili, že toto maximum nastává přibližně při třetinové frekvenci otáčení míchadla, která je potřebná k~dosažení vznosu částic (tedy $\frac{1}{3}N_{js}$).

Výše zmíněné studie se příliš nezabývaly náběhovou fází suspendace, což je přechod od nehybné vrstvy částic do ustáleného stavu systému. Právě na numerickou simulaci tohoto děje se zaměřila skupina autorů \citet{tamb09}. Experimentální aparatura se skládala z~válcové nádoby o~průměru $T=\SI{0.19}{\meter}$ a jenž obsahovala čtyři radiální narážky o~šířce $b=T/10$. Vsádku tvořila voda a částice oxidu křemičitého o~celkové koncentraci \volproc{9.6}. K~míchání této suspenze byla využita Rushtonova turbína, která byla umístě ve vzdálenosti $C=\SI{0.017}{\meter}$ ode dna nádoby. Pro zachycení průběhu suspendace byla využita digitální kamera s~vysokým rozlišením schopna vyhotovit \num{19} snímků za sekundu. Všechny CFD simulace byly provedeny s~využít vícefázového přístupu Eulerian-Eulerian. Turbulence byla modelována pomocí homogenního \keps{} modelu se zohledněním turbulentní disperze. Autoři použili několik způsobů vyjádření koeficientu odporu pro popis mezifázové odporové síly. Jednalo se o~korelace které navrhli výzkumníci \citet{cli78}, \citet{bru98} a konstantě zvolenou hodnotu \num{6.01}. Vytvořená výpočetní síť obsahovala \num{52992} buněk, přičemž pro simulaci rotace míchadla byla využita technika klouzající sítě. Ze získaných výsledků vyplynulo, že suspendace pevné fáze v~mechanicky míchané nádobě je především řízena gravitační, setrvačnou a odporovou silou. Pokud jsou tyto síly vhodně modelovaný, tak lze dosáhnout uspokojivé shody s~experimentálními údaji, i při popisu přechodového chování suspenze. Brucatova korelace nejlépe korespondovala s~experimentálními výsledky než zbylé dva přístupy. Při vyšších rychlostech otáčení míchadla se rozdíly mezi těmito modely začínaly vytrácet a po dosažení frekvence otáčení \SI{20}{\per\second} byly již téměř zanedbatelné.

\citet{hos10} studovali vliv typu míchadla, jeho světlé výšky a rychlosti otáčení na stupeň homogenizace směsi kapalina-pevná fáze v~mechanicky míchané nádobě. Tato nádoba byla tvořena průhlednou válcovou nádrží s~plochým dnem o~vnitřním průměru $T=\SI{0.3}{\meter}$, jenž byla navíc opatřena čtyřmi radiálními narážkami. Jako vsádka byla použita voda spolu se skleněnými kuličkami o~průměru od \SI{100}{\micro\meter} do \SI{900}{\micro\meter}. K~jejímu promíchávání byly využity tři míchadla typu Lightnin (A100, A200 a A310) a jejich výška ode dna byla měněna v~rozmezí $T/6$ až $T/2$. Distribuce pevné fáze byla experimentálně stanovena pomocí elektrické odporové tomografie, přičemž po obvodu nádoby bylo celkem rozmístěno \num{128} čidel. Pro numerickou simulaci autoři použili vícefázový model Eulerian-Eulerian spolu se standardním se standardním \keps{} modelem pro turbulenci. Dále vytvořili pro každé míchadlo nestrukturovanou výpočetní síť, jenž v~průměru obsahovala \num{315000} čtyřstěnných buněk. Ve všech simulačních výpočet byla využita empirická korelace pro koeficient odporu navržená Gidaspowem (1994). Frekvence otáčení míchadla potřebná k~dosažení vznosu částic stanovená pomocí CFD byla blízko hodnotě určené dle Zwieteringova vztahu. Dále bylo dle očekávání  zjištěno, že jakmile stupeň homogenizace dosáhne svého maxima, tak již další zvyšování otáčení míchadla resp. jeho příkonu nemá vliv na kvalitu suspendace. V~CFD simulacích bylo nejvyššího stupně homogenizace dosaženo při světlé výšce míchadla $C=T/3$, což je v~dobré shodě s~provedeným experimenty. Po porovnání získaných výsledků pro jednotlivá míchadla vyplynulo, že míchadlo typu A100 dosazuje při stejných podmínkách vyššího stupně homogenizace než zbylé dva modely.

Výzkumníci \citet{sar10} zkoumali hysterezi výšky suspenzního mraku během suspendace pevné fáze v~mechanicky míchané nádobě. 
Proto tento jev je charakteristické, že výška vznosu částic se liší podle toho, zda frekvence otáčené míchadla byla v~minulosti zvyšována, či snižována. Ke studiu toho jevu autoři využili, jak svá vlastní experimentální měření, tak techniku CFD. Experimentální aparatura byla tvořena válcovou nádrži se čtyřmi narážkami o~průměru $T=\SI{0.7}{\meter}$. Do této nádoby bylo umístěno šestilopatkové míchadlo se šikmo skloněnými lopatkami, přičemž jeho světlá výška činila $C=T/3$. Vsádka se skládala z~vody a skleněných částic o~průměru \SI{50}{\micro\meter} a \SI{250}{\micro\meter}. Frekvence otáčení míchadla byla nastavována v~intervalu od \SI{2}{\per\second} do \SI{10}{\per\second} a maximální koncentrace pevné fáze byla zvolena \volproc{7}. K~experimentálnímu stanovení výšky vznosu pevné fáze autoři využili metody obrazové analýzy při které pořizovali záznam průběhu suspendace pevné fáze v~nasvícené nádobě. Navíc experimentátoři také měřili axiální rychlost částic v~nádobě pomocí ultrazvukového čidla umístěného rovnoběžně s~osou míchadla. Vlastní numerická simulace byla založena na vícefázovém přístupu Eulerian-Granular a standardním turbulentním modelu \keps{}. Vytvořená výpočetní doména obsahovala celkem \num{501746} šestistěnných buněk. Pro popis odporového koeficientu byly využity vztahy, jenž navrhli \citet{bru98} a \citet{kho06}. Z~naměřených výsledků vyplynulo, že při nižších koncentracích pevné fáze výška suspenzního mraku vykazovala monotonií průběh v~závislosti na rychlosti otáčení míchadla. V~oblasti vyšších koncentrací (\volproc{5} a \volproc{7}) však s~rostoucí frekvencí otáčení míchadla výška mraku nejprve klesala, a až poté začala růst. Autorům se podařilo experimentálně pozorovat hysterezi výšky suspenzního mraku především při množství pevné fáze rovnu \volproc{5}. Toto zjištění posléze úspěšně potvrdily provedené CFD simulace. Navíc model pro koeficient odporu podle Khopkara vykazoval lepší shodu s~experimentální výsledky než korelace navržena Brucatem.


