\chapter{Literární rešerše}
Suspendace pevné fáze v mechanicky míchaných nádobách patří v posledních desetiletích ke značně studovanému jevu. Za průkopníka v této oblasti je označován \citet{zwi58}, jenž vyvinul empirickou korelaci pro výpočet frekvence otáčení míchadla při které již dochází ke vznosu částic. Při této frekvenci žádná částice nezůstává na dně nádoby déle než jednu či dvě sekundy. Také další autoři intenzivně experimentálně studovali mechanizmus suspendace v promíchávaných nádobách \citep{nie68,bal78,arm98}. 

V posledních třech dekádách mnoho výzkumníků obrátilo svoji pozornost k počítačové dynamice tekutin jako k prostředku ke zkoumání procesů probíhajících ve vícefázových systémech. Od té doby technika CFD nabývá na vzrůstající oblibě jako základního nástroje ke studiu tokového pole a distribuce pevných částic v míchaných nádobách. Rané CFD simulace byly pouze zaměřeny na určení rychlostního pole v jednofázových systémech \citep{kre91}. Experimentální metody jako LDV byly používány k určení nutných okrajových podmínek pro modelování pohybu míchadla. Hlavní nevýhodou takto získaných simulací je především jejich úzký vztah se zkoumaným systémem. Další vývoj na poli CFD se zaměřil na vývoj simulačních metod, které by kompletně odstranili tuto závislost na experimentálních datech. Tyto snahy vyústily k tvorbě komplexních technik jako MRF a SM, které jsou schopny popsat rotaci míchadla bez dalších dodatečných dat.  

Zatímco jednofázové toky v míchaných nádobách byly podrobně zkoumány, jak experimentálně, tak výpočetně. Mnohem méně prací bylo publikováno na téma vícefázového proudění v míchaných systémech. Následující kapitola shrnuje několik významných prací, který byly provedeny za účelem studie suspendace pevné fáze v mechanicky promíchávaných nádobách.
  
\section{Suspendace pevné fáze v míchaných nádobách}
\citet{lju01} simulovali dvoufázové proudění v nádobě se šestilopatkovým míchadlem se šikmo skloněnými lopatkami. Zkoumaný systém byl tvořen válcovou nádobou s plochým dnem a čtyřmi radiálními narážkami. Vnitřní průměr této nádoby činil $T=\SI{0.297}{\meter}$ a jako vsádka byla použita voda. Zrnitou fázi tvořily skleněné částice o průměru od \SI{150}{\micro\meter} do \SI{450}{\micro\meter}. Pro simulaci vícefázového proudění autoři zvolil techniku Eulerian-Eulerian spolu se standardním \keps{} turbulentním modelem. Výpočetní doména se přibližně skládala ze \num{52000} buněk a díky symetrii systému, pouze čtvrtina nádoby byla simulována. K popisu mezifázové odporové síly výzkumníci využil čtyři odlišné korelace pro koeficient odporu. Tyto modely navrhli autoři: \citet{schi32}, \citet{ish79}, \citet{ihme72} a \citet{bru98}. Získaná simulační data byla porovnána s experimentálními údaji naměřenými pomocí fázové Dopplerové anemometrie. V získaných výsledcích se mezifázový odpor projevil jako dominantní silou působící na částice pevné fáze. Na druhou stranu virtual mass force a lift force měla po celou dobu zanedbatelný vliv na výsledky simulace. Člen mezifázové turbulentní disperze především ovlivňoval řešení v blízkosti osy nádoby, avšak ve zbytku systému byl jeho působení zanedbatelné. Závěrem autoři konstatovali, že všechny použité modely pro koeficient odporu poskytovali velmi podobné výsledky.          

Solid-liquid simulations of the turbulent flow according to the Eulerian-Lagrangian method were carried out by \citet{derk03}. The author employed large eddy simulation to model suspension of solids in a stirred vessel, which contained a Rushton turbine. The stirred tank had a standard configuration. It consisted of a cylindrical, flat-bottomed, baffled tank with diameter $T=\SI{0.297}{\meter}$. The working fluid was water, whereas glass particles of diameter were used as the solid phase. It is worth noting that concentration of the solids was relatively low (up to \volproc{3.6}). Instead of commercial CFD solvers the author used his own CFD code based on Lattice-Boltzmann method. Simulation results qualitatively corresponded to experimental data provided by \citet{miche03} for the similar solid-liquid system. The absence of the lift and the virtual mass force had no significant impact on concentration profiles. It was observed that only if particle-particle collisions were taken into account, a realistic particle distribution throughout the tank was obtained.



Suspension height is the one of important characteristics of mixed systems. Researchers \citet{mic04} carried out experimental measurements of the suspension height at various agitation speeds in a fully baffled transparent tank. The vessel diameter was $T=\SI{0.19}{\meter}$ and the liquid level was chosen higher then usual to facilitate observation of the particle free region. Silica spheres in the size range \num{212}--\SI{250}{\micro\meter} were used as the solid phase and their amount varied from \volproc{0.45} to \volproc{14.4}. Also CFD simulations of the described system were performed in order to test their ability to properly predict the formation of a clear liquid layer. The Eulerian-Eulerian multiphase model was adopted in conjunction with the sliding mesh method for a Rushton turbine motion. Due to symmetric nature of the system, only half of the vessel was simulated and the created computational domain was made up of \num{52992} hexahedral cells. High turbulent flow was modeled using homogeneous \keps{} approach, where both phases are assumed to share the same values for turbulent kinetic energy and turbulent dissipation. In this work drag coefficient was computed on the basis of well known \citet{schi32} correlation. The conducted experiments showed almost linear dependence between the suspension height and the impeller speed. From obtained results arose that the Eulerian-Eulerian model is able to qualitatively predict the height of suspension cloud. The authors suggested taking particle-particle interactions and free turbulence effect into account to obtain better agreement with experimental results.                 
