\chapter{Literární rešerše}
Suspendace pevné fáze v mechanicky míchaných nádobách patří v posledních desetiletích ke značně studovanému jevu. Za průkopníka v této oblasti je označován \citet{zwi58}, jenž vyvinul empirickou korelaci pro výpočet frekvence otáčení míchadla při které již dochází ke vznosu částic. Při této frekvenci žádná částice nezůstává na dně nádoby déle než jednu či dvě sekundy. Také další autoři intenzivně experimentálně studovali mechanizmus suspendace v promíchávaných nádobách \citep{nie68,bal78,arm98}. 

V posledních třech dekádách mnoho výzkumníků obrátilo svoji pozornost k počítačové dynamice tekutin jako k prostředku ke zkoumání procesů probíhajících ve vícefázových systémech. Od té doby technika CFD nabývá na vzrůstající oblibě jako základního nástroje ke studiu tokového pole a distribuce pevných částic v míchaných nádobách. Rané CFD simulace byly pouze zaměřeny na určení rychlostního pole v jednofázových systémech \citep{kre91}. Experimentální metody jako LDV byly používány k určení nutných okrajových podmínek pro modelování pohybu míchadla. Hlavní nevýhodou takto získaných simulací je především jejich úzký vztah se zkoumaným systémem. Další vývoj na poli CFD se zaměřil na vývoj simulačních metod, které by kompletně odstranili tuto závislost na experimentálních datech. Tyto snahy vyústily k tvorbě komplexních technik jako MRF a SM, které jsou schopny popsat rotaci míchadla bez dalších dodatečných dat.  

Zatímco jednofázové toky v míchaných nádobách byly podrobně zkoumány, jak experimentálně, tak výpočetně. Mnohem méně prací bylo publikováno na téma vícefázového proudění v míchaných systémech. Následující kapitola shrnuje několik významných prací, který byly provedeny za účelem studie suspendace pevné fáze v mechanicky promíchávaných nádobách.
  
\section{Suspendace pevné fáze v míchaných nádobách}
\citet{lju01} simulovali dvoufázové proudění v nádobě se šestilopatkovým míchadlem se šikmo skloněnými lopatkami. Zkoumaný systém byl tvořen válcovou nádobou s plochým dnem a čtyřmi radiálními narážkami. Vnitřní průměr této nádoby činil $T=\SI{0.297}{\meter}$ a jako vsádka byla použita voda. Zrnitou fázi tvořily skleněné částice o průměru od \SI{150}{\micro\meter} do \SI{450}{\micro\meter}. Pro simulaci vícefázového proudění autoři zvolil techniku Eulerian-Eulerian spolu se standardním \keps{} turbulentním modelem. Výpočetní doména se přibližně skládala ze \num{52000} buněk a díky symetrii systému, pouze čtvrtina nádoby byla simulována. K popisu mezifázové odporové síly výzkumníci využil čtyři odlišné korelace pro koeficient odporu. Tyto modely navrhli autoři: \citet{schi32}, \citet{ish79}, \citet{ihme72} a \citet{bru98}. Získaná simulační data byla porovnána s experimentálními údaji naměřenými pomocí fázové Dopplerové anemometrie. V získaných výsledcích se mezifázový odpor projevil jako dominantní silou působící na částice pevné fáze. Na druhou stranu virtual mass force a lift force měla po celou dobu zanedbatelný vliv na výsledky simulace. Člen mezifázové turbulentní disperze především ovlivňoval řešení v blízkosti osy nádoby, avšak ve zbytku systému byl jeho působení zanedbatelné. Závěrem autoři konstatovali, že všechny použité modely pro koeficient odporu poskytovali velmi podobné výsledky.

Another way how to simulate suspension of solids is using the Eulerian-Granular multiphase model. This technique was employed by \citet{oshi02} to study the distribution of solids in a stirred tank, where concentration of particles varied from \volproc{0.5} to \volproc{50}. In addition, the authors determined quality of suspension (i.e. standard deviation of concentration), its height and the mixing time in the liquid phase. The investigated system was a cylindrical vessel with a four-bladed pitched blade turbine. Due to simplicity the computational grid was set up as two-dimensional axisymmetric model, which reduced computation time by a significant amount. Turbulence was modeled using the standard \keps{} approach and the granular viscosity was expressed by a correlation designed by \citet{syam93}. The obtained results were in the good agreement with experimental data from literature. However, the inconsistency between the just suspended speed correlation and $N_{js}$ for the evaluated tank was observed. The authors suggested that the difference could be caused by high efficiency of modern impellers together with their low off-bottom clearances during the experiments. Nevertheless, the standard deviation of solids volume fraction was shown to be useful measure of the quality of suspension. 

\citet{derk03} provedl sérii simulací kapalina a pevná fáze pomocí matematického modelu Eulerian-Lagrangian. Navíc autor využil techniku velkých vírů (LES) k popisu suspendace v mechanicky míchané nádobě pomocí Rushtonovy turbíny. Míchací nádoba měla standardní konfiguraci, jenž byla tvořena válcovou nádrží s plochým dnem o průměru $T=\SI{0.297}{\meter}$. Jako pracovní médium byla použita voda do které byly ponořeny částice pevné fáze o velikosti \SI{300}{\micro\meter}. Za zmínku stojí fakt, že ve všech provedených simulacích byla koncentrace pevné fáze relativně malá (do \volproc{3.6}). Získané výsledky kvalitativně korespondovali s experimentálním měřením, které provedl \citet{miche03} pro podobný systém kapalina-pevná fáze. Absence lift and the virtual mass force neměla žádný významný dopad na výsledné koncentrační profily. Autor pozoroval, že pouze pokud jsou zahrnuty do simulace interakce mezi jednotlivými částicemi, tak distribuce pevné fáze v nádobě má realistický charakter.

Výška suspenze je jednou z významných charakteristik promíchávaných systémů. Autoři \citet{mic04} provedli sérii experimentálních měření výšky vznosu pevné fáze pro různé rychlosti otáčení míchadla v průhledné nádobě. Průměr nádoby činil $T=\SI{0.19}{\meter}$ a hladina kapaliny byla vyšší než obvykle, aby se usnadnilo pozorování oblasti ve které se nenacházejí žádné částice. Kuličky oxidu křemičitého o velikosti v rozmezí num{212}--\SI{250}{\micro\meter} byly použity jako pevná fáze a jejich množství se lišilo od \volproc{0.45} to \volproc{14.4}. Kromě experimentů byly provedeny simulace výše popsaného systému za účelem zjištění schopnosti predikce tohoto jevu pomocí CFD. V práci byl použit vícefázový model Eulerian-Eulerian spolu s technikou klouzající sítě pro popis pohybu Rushtonovy turbíny. Díky symetrické povaze úlohy pouze jedna polovina nádoby byla simulována a výsledná výpočetní doménu tvořilo \num{52992} šestistěnných buněk. Vysoce turbulentní proudění bylo modelováno pomocí homogenního \keps{} přístupu, při kterém obě fáze sdílejí stejné hodnoty turbulentní kinetické energie a rychlosti disipace turbulentní energie. V této práci byla použita dobře známá korelace pro odporový koeficient, kterou navrhli \citet{schi32}. Provedené experimenty ukázaly téměř lineární závislost mezi výškou suspenzního mraku a frekvencí otáčení míchadla. Ze získaných výsledků vyplynulo, že model Eulerian-Eulerian je schopen kvalitativně popsat výšku vznosu zrnité fáze. Na závěr autoři doporučili, pro zlepšení shody s experimentálními výsledky, zohlednění turbulence v modelu pro koeficient odporu a zahrnutí interakcí mezi jednotlivými částicemi.  