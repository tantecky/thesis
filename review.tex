\chapter{Literature review}
The suspension of solids has been investigated very extensively over the years. Pioneering was made by \citet{zwi58}, who provided an empirical dimensionless correlation for the  stirred speed at which solids are just suspended. For this velocity is characteristic that no particle remains stationary on the bottom of a tank for more than one or two seconds. Also other authors have experimentally studied the mechanism of suspension of solids in agitated vessels \citep{nie68,bal78,arm98}.   

In the last two decades many researchers have turned their attention to computational fluid dynamics in order to gain better understanding of the phenomena that govern the multiphase systems. Since then CFD has been increasingly employed as an essential tool to minutely analyze the entire flow field and the solid particle distribution. Early CFD simulations were focused only on obtaining a velocity field for the single-phase flow in stirred tanks \citep{kre91}. Experimental methods like LDV  were being used to provide necessary boundary conditions for the modelling of the impellers. The main disadvantage of simulations supported by LDV measurements was that obtained results were very closely connected with the particular experiment. Further research on the field of CFD was aimed to create simulation methods which would completely remove the dependency on experimental data. These efforts have resulted in the development of complex modelling techniques such as the multiple reference frames method or the sliding mesh method, which are able to properly describe the motion of the impellers without the need of additional data.  

While the single-phase flow in agitated vessels has been investigated thoroughly both experimentally and computationally, much less work has been published on the two-phase flow. The following paragraphs

              

\section{Particle drag coefficient}

