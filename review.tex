\chapter{Literature review}
The suspension of solids has been investigated very extensively over the years. Pioneering was made by \citet{zwi58}, who provided an empirical dimensionless correlation for the  stirred speed at which solids are just suspended. For this velocity is characteristic that no particle remains stationary on the bottom of a tank for more than one or two seconds. Also other authors have experimentally studied the mechanism of suspension of solids in agitated vessels \citep{nie68,bal78,arm98}.   

In the last three decades many researchers have turned their attention to computational fluid dynamics in order to gain better understanding of the phenomena that govern the multiphase systems. Since then CFD has been increasingly employed as an essential tool to minutely analyze the entire flow field and the solid particle distribution. Early CFD simulations were focused only on obtaining a velocity field for the single-phase flow in stirred tanks \citep{kre91}. Experimental methods like LDV  were being used to provide necessary boundary conditions for the modelling of the impellers. The main disadvantage of simulations supported by LDV measurements was that obtained results were very closely connected with the particular experiment. Further research on the field of CFD was aimed to create simulation methods which would completely remove the dependency on experimental data. These efforts have resulted in the development of complex modelling techniques such as the multiple reference frames method or the sliding mesh method, which are able to properly describe the motion of the impellers without the need of additional data.  

While the single-phase flow in agitated vessels has been investigated thoroughly both experimentally and computationally, much less work has been published on the two-phase flow. The following chapecter summarizes several important computational and experimental studies that were carried out to investigate solid-liquid flow in stirred vessels.  
  

\section{Solid-liquid flow in stirred tanks}
\citet{lju01} simulated the two-phase flow in a vessel agitated by a pitched blade turbine with four blades. The investigated system consisted of a flat-bottomed tank equipped with four equally spaced baffles. A diameter of the vessel was $T=\SI{0.297}{\meter}$ and it was filled with water. Glass particles of diameter \SI{150}{\micro\meter} and \SI{450}{\micro\meter} were used as the solid phase. The authors applied Eulerian-Eulerian approach coupled with the standard \keps{} model for their multiphase simulations. A computational domain contained approximately \num{52000} cells and due to the symmetry of the system, only one quarter of the tank was considered. For the description of the interphase drag force four different correlations of drag coefficient were employed, namely \citet{schi32}, \citet{ish79}, \citet{ihme72} and \citet{bru98}. Moreover the authors estimated influence of additional non drag forces on obtained results. The acquired simulation data were compared with experimental measurements done by phase-Doppler anemometry (PDA). It was found that interphase drag is the dominating force acting on the particles. On the other hand the virtual mass force and the lift force had minor impact on the solution. The turbulent dispersion term exhibited some effect in the region close to the center of the vessel, but in the rest of the tank its effect was negligible. All the drag models, which were tested, gave similar very similar results.

Another way how to simulate suspension of solids is using the Eulerian-Granular multiphase model. This technique was employed by \citet{oshi02} to study the distribution of solids in a stirred tank, where concentration of particles varied from \volproc{0.5} to \volproc{50}. In addition, the authors determined quality of suspension (i.e. standard deviation of concentration), its height and the mixing time in the liquid phase. The investigated system was a cylindrical vessel with a four-bladed pitched blade turbine. Due to simplicity the computational grid was set up as two-dimensional axisymmetric model, which reduced computation time by a significant amount. Turbulence was modeled using the standard \keps{} approach and the granular viscosity was expressed by a correlation designed by \citet{syam93}. The obtained results were in the good agreement with experimental data from literature. However, the inconsistency between the just suspended speed correlation and $N_{js}$ for the evaluated tank was observed. The authors suggested that the difference could be caused by high efficiency of modern impellers together with their low off-bottom clearances during the experiments. Nevertheless, the standard deviation of solids volume fraction was shown to be useful measure of the quality of suspension. 

Solid-liquid simulations of the turbulent flow according to the Eulerian-Lagrangian method were carried out by \citet{derk03}. The author employed large eddy simulation to model suspension of solids in a stirred vessel, which contained a Rushton turbine. The stirred tank had a standard configuration. It consisted of a cylindrical, flat-bottomed, baffled tank with diameter $T=\SI{0.297}{\meter}$. The working fluid was water, whereas glass particles of diameter were used as the solid phase. It is worth noting that concentration of the solids was relatively low (up to \volproc{3.6}). Instead of commercial CFD solvers the author used his own CFD code based on Lattice-Boltzmann method. Simulation results qualitatively corresponded to experimental data provided by \citet{miche03} for the similar solid-liquid system. The absence of the lift and the virtual mass force had no significant impact on concentration profiles. It was observed that only if particle-particle collisions were taken into account, a realistic particle distribution throughout the tank was obtained.

Suspension height is the one of important characteristics of mixed systems. Researchers XXX carried out experimental measurements of the suspension height at various agitation speeds in a fully baffled transparent tank. The vessel's diameter was $T=\SI{0.19}{\meter}$ and the liquid level was chosen higher then usual to facilitate observation of the particle free region. Silica spheres in the size range \num{212}--\SI{250}{\micro\meter} were used as the solid phase and their amount varied from \volproc{0.45} to \volproc{14.4}. Also CFD simulations of the described system were performed in order to test their ability to properly predict the formation of a clear liquid layer.        