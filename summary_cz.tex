\section*{Souhrn}
Následující práce se zabývá simulací procesu suspendace v~mechanicky míchané nádobě pomocí metody počítačové dynamiky tekutin. Hlavním cílem bylo stanovit výšku sus\-pen\-zní\-ho mraku a posoudit vliv modelu pro koeficient odporu na distribuci pevné fáze. Zkoumaný systém se skládal z~válcové nádoby s~plochým dnem o~vnitřním průměru $T=\SI{0.29}{\meter}$. Pevnou fázi tvořily kuličky z~PVC o~koncentraci 5, 10 a \volproc{15} a jako kapalná vsádka byla použita voda a polyvinylpyrrolidon. Výška plnění byla zvolena $H=T$ a k~promíchání bylo použito šestilopatkové míchadlo se šikmo skloněnými lopatkami (úhel zkosení \SI{45}{\degree}). K~vlastní simulaci byl využit komerční software \flu{} 12.1.4, do kterého bylo implementováno několik modelů pro koeficient odporu v~podobě uživatelem definovaných funkcí. Pro popis vícefázového proudění byly použity přístupy Eulerian-Eulerian a Eulerian-Granular spolu se standardním \keps{} turbulentním modelem. Výsledky získané numerickou simulací dobře korespondovaly s~experimentálním měřením. Zvláště dobré shody bylo dosaženo především v~případech, kdy objemový zlomek pevné fáze činil \proc{5} a \proc{10}.
