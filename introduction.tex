\chapter{Úvod}
Proces míchaní patří k~nejběžnějším jevům vyskytujícím se v~průmyslové praxi. Druhým nejčastějším typem míchací operace je promíchávání kapalné a pevné fáze, jenž zahrnuje např. suspendaci, dispergaci pevných částic, rozpouštění, krystalizaci nebo sus\-pen\-zní polymeraci. Provozní podmínky u~těchto procesů obecně záleží na požadovaném stavu, kterého má být dosaženo. Pro jejich určení je třeba zohlednit všechny fyzikální a chemické jevy nutné k~dosažení žádaného provozního výsledku. 

Podstatou suspendace je udržení pevné fáze ve vznosu tak, aby se zabránilo jejímu usazování a zlepšil se mezifázový prostup hmoty či tepla. Pokud je zvolená úroveň suspendace nedostatečná (např. v~důsledku nízké frekvenci otáčení míchadla), tak kvalita výsledných produktů může být ohrožena. Naopak příliš vysoký stupeň homogenizace směsi nemusí mít praktický přínos a tudíž představuje pouze zbytečné navýšení provozních výdajů. Hlavním cílem při návrhu míchacích aparátů je tedy dosažení dostatečného stavu suspendace při minimálních nákladech. 

Ke stanovení provozních parametrů míchacích aparátů se často využívají empirické vzorce vytvořené na základně experimentů. Problém však může nastat, pokud jsou provozní po\-ža\-dav\-ky příliš atypické a neexistují k~nim vhodná experimentální data. Jedním z~dalších možných přístupů, jak modelovat děje pro\-bí\-ha\-jí\-cí v~promíchávaném systému, je využití metody po\-čí\-ta\-čo\-vé dynamiky tekutin (CFD). 
Ve svých počátcích byla technika CFD využívána v~leteckém průmyslu pro simulaci proudění vzduchu podél náběžné hrany křídla. 
S~prudkým rozvojem výpočetní techniky postupně docházelo k~tvorbě komplexnějších modelů a algoritmů, které jsou schopny se blíže přiblížit reálnému chování zkoumané soustavy, avšak za cenu rostoucích hardwarových nároků. V~sou\-čas\-nos\-ti se technika CFD využívá ve všech oborech zabývajících se prouděním tekutin, kde experimentální měření by bylo příliš nákladné nebo obtížně realizovatelné.   

Hlavním cílem předkládané práce bylo právě využití metody počítačové dynamiky tekutin pro určení distribuce pevné fáze v~mechaniky míchané nádobě spolu se stanovením výšky suspenzního mraku. Dále byly zkoumány rozdíly mezi jednotlivými simulacemi při použití odlišných modelů pro výpočet koeficientu odporu.  Získané výsledky byly porovnány s~dostupnými experimentálními daty, jenž naměřila \citet{pav11}.
