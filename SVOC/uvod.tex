\chapter{Úvod}
Proces míchaní patří k~nejběžnějším jevům vyskytující se v~průmyslové praxi. Druhým nejčastějším typem míchací operace je promíchávání kapalné a pevné fáze, jenž zahrnuje např. suspendaci, dispergaci pevných částic, rozpouštění, krystalizaci nebo suspenzní polymeraci. Během suspendace se udržuje pevná fáze ve vznosu, aby se zabránilo jejímu usazování a zlepšil se mezifázový prostup hmoty či tepla. 

Při návrhu míchacích aparátů se často vychází z~empirických vzorců vytvořených na základně experimentů. Problém však může nastat pokud jsou provozní požadavky příliš atypické a neexistují k~nim vhodná experimentální data. Jedním z~dalších možných přístupů, jak modelovat děje probíhající v~promíchávaném systému, je využití metody počítačové dynamiky tekutin (CFD). V~současnosti se technika CFD využívá ve všech oborech zabývající se prouděním tekutin, kde experimentální měření by bylo příliš nákladné nebo obtížně realizovatelné.   
