\chapter{Závěr}
Následující práce byla zaměřena na posouzení vlivu použitého modelu pro koeficient odporu na výsledky CFD simulace suspendace v~mechanicky míchané nádobě. V~rámci této studie bylo implementováno několik korelací pro koeficient odporu v~turbulentní oblasti proudění pomocí uživatelsky definovaných funkcí. Pomocí techniky CFD bylo simulováno promíchávání pevné fáze o~koncentraci 5\,\%\,obj. ve vsádce polyvinylpyrrolidonu.

Ze získaných výsledků vyplynulo, že daný model pro koeficient odporu ovliňuje distribuci pevné fáze zvláště při vyšších objemových zlomcích dané fáze. Tento fakt dobře ilustruje obr. \ref{fig:vol2}. Model navržený \hyperlink{hyp:cds}{Brucatem} pro Taylorův–Couetteovův tok vykazoval ve všech případech výrazně odlišné chování, a proto ho nelze doporučit pro simulaci suspendaci v~mechanicky míchané nádobě. Zbývající modely projevovaly velmi podobné chování, co se týče výsledných koncentračních profilů. 

Budoucí výzkum se zaměří na stanovení výšky vznosu pevné fáze spolu s~určením doby homogenizace a porovnáním s~dostupnými experimentálními výsledky. CFD simulace bude provedena i pro vyšší objemové zlomky pevné fáze spolu se srovnáním dalších vícefázových modelů (např. \nameref{sec:egm}).  
