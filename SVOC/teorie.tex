\chapter{Teoretická část}

\section{Vícefázové modely}
V~současnosti existuje řada matematických modelů, které popisují vícefázové proudění. Následující kapitola obsahuje přehled modelů, které jsou vhodné k~simulaci suspendace v~mechanicky míchaných nádobách.

\subsection{Eulerian-Lagrangian}
Tento typ modelu uvažuje primární tekutou fází jako kontinuum s~dispergovanou sekundární fází. Pro primární fázi je řešena rovnice kontinuity spolu s~Navierovými-Stokesovými rovnicemi, zatímco pro dispergovanou fázi je řešena trajektorie každé částice separátně. Jednotlivé fáze si mohou mezi sebou vyměňovat hmotu, hybnost a energii, avšak vzájemné interakce částic nebo jejich rozpad jsou zanedbány. Model Eulerian-Lagrangian je především vhodný pro systémy, kde objemový zlomek dispergované fáze nepřesáhne \SI{10}{\percent} např: rozprašovací sušárny, cyklóny nebo spalování uhlí či kapalného paliva. 

Řešením bilance sil působící na částici (rov. \ref{eq:dpm}) je získána její trajektorie v~daném časovém okamžiku.

\begin{equation}
	m_{p}\frac{d\vec{v}_{p}}{dt} = \vec{F}_{D} + \vec{F}_{g} + \vec{F}_{vz} + \vec{F}_{ad}
	\label{eq:dpm}
\end{equation} 

\noindent Kde jednotlivé členy značí:

\begin{itemize}[itemsep=0pt,parsep=0pt,partopsep=0pt,topsep=0pt]
  \item $m_{p}$ hmotnost částice
  \item $\vec{v}_{p}$ rychlost částice
  \item $\vec{F}_{D}$ odporovou sílu
  \item $\vec{F}_{g}$ gravitační sílu
  \item $\vec{F}_{vz}$ vztlakovou sílu
  \item $\vec{F}_{ad}$ další síly (např: tlaková, zdánlivá síla, síla zahrnující vliv rotace atd.)
\end{itemize}

\subsection{Eulerian-Eulerian}
U~modelu Eulerian-Eulerian jsou jednotlivé fáze považovány za prostupující se kontinua a každý bod v~systému obsahuje informaci o~objemovém zlomku dané fáze. Z~tohoto popisu je zřejmé, že suma objemových zlomků přes všechny fáze v~libovolném bodě se vždy musí rovnat jedné. 

\begin{equation}
	\sum_{i=1}^n \alpha_{i} = 1
	\label{eq:volfrac}
\end{equation} 

\noindent Jednotlivé fáze mohou být kapalné, plynné nebo pevné a jejich celkový počet není teoreticky limitován. Pro každou fázi se řeší rovnice kontinuity (\ref{eq:conti}) a sada rovnic pro hybnost. (\ref{eq:moml}). K~výměně hybnosti mezi jednotlivými fázemi slouží mezifázové členy v~těchto rovnicích. Pokud dochází k~přenosu tepla nebo hmoty je třeba tuto skutečnost zohlednit v~bilanci energie a hmoty.    

Když probíhá výměna hmoty mezi jednotnými fázemi tak rovnice kontinuity pro $i$-tou fázi má tvar:

\begin{equation}
	\frac{\partial}{\partial t} (\alpha_{i}\rho_{i}) +  \nabla \cdot (\alpha_{i}\rho_{i}\vec{v}_{i}) = 0
	\label{eq:conti}
\end{equation}

\noindent Rovnice hybnosti pro $i$-tou fázi za stejných předpokladů lze zapsat jako:

\begin{equation}
	\frac{\partial}{\partial t} (\alpha_{i}\rho_{i}\vec{v}_{i}) + \nabla \cdot (\alpha_{i}\rho_{i} \vec{v}_{i} \otimes \vec{v}_{i}) = -\alpha_{i} \nabla p + \nabla \cdot \bar{\tau}_{i} + \alpha_{i}\rho_{i}\vec{g} + \sum_{j=1}^n \vec{R}_{ji} + \vec{F}_{ext} + \vec{F}_{int}
	\label{eq:moml}
\end{equation}

\noindent kde $p$ je tlak, $\bar{\tau}_{i}$ je deviátor tenzoru napětí, jehož konkretní tvar závisí na typu uvažované fáze, a $\alpha_{i}\rho_{i}\vec{g}$ má význam gravitační síly působící na objemový element. Člen $\vec{R}_{ji}$ představuje mezifázovou odporovou sílu mezi $i$-tou a $j$-tou fází, $\vec{F}_{ext}$ má význam dalších objemových sil a $\vec{F}_{int}$ zahrnuje povrchové síly působící na $i$-tou fázi vlivem ostatních fází. 

\subsection{Eulerian-Granular}
\label{sec:egm}
Model Eulerian-Granular se liší od předchozího tím, že  popis chování pevné fáze byl odvozen s~využitím kinetické teorie, která je například známá ze statistického popisu plynů. U~tohoto modelu se viskozita pevné fáze mění v~závislosti na interakcích s~ostatními částice a primární fází. Na rozdíl od modelu Eulerian-Eulerian model Eulerian-Granular umožňuje nastavit maximální hodnotu objemového zlomku pevné fáze, která je pro sférické částice typicky kolem hodnoty \num{0.6}. Mezi nejčastější aplikace patří simulace fluidních loží nebo suspendace v~mechanicky míchaných nádobách.

Bilanci hybnosti pro pevnou fázi $s$ má tvar:

\begin{equation}
	\frac{\partial}{\partial t} (\alpha_{s}\rho_{s}\vec{v}_{s}) + \nabla \cdot (\alpha_{s}\rho_{s} \vec{v}_{s} \otimes \vec{v}_{s}) = -\alpha_{s} \nabla p - \nabla p_{s} + \nabla \cdot \bar{\tau}_{s} + \alpha_{s}\rho_{s}\vec{g} + \sum_{j=1}^n \vec{R}_{ji} + \vec{F}_{ext} + \vec{F}_{int}
	\label{eq:momls}
\end{equation}
  
\noindent Člen $p_{s}$ je tlak pevné fáze, který je důsledkem neelastických srážek mezi částicemi, a $\bar{\tau}_{s}$ je deviátor tenzoru napětí pro pevnou fázi.       

\section{Mezifázová odporová síla}

Člen mezifázové odporové síly $\vec{R}_{ji}$ v~rovnici \ref{eq:moml} nebo \ref{eq:momls} nejvýznamněji přispívá do popisu interakce mezi jednotlivými fázemi, a proto správnost popisu tohoto členu zásadně ovlivňuje kvalitu výsledné simulace. 

Tuto sílu lze rozepsat jakou součin koeficientu mezifázového sdílení hybnosti a relativní rychlosti $i$-té a $j$-té fáze.

\begin{equation}
	\sum_{j=1}^n \vec{R}_{ji} = K_{ji} \left( \vec{v}_{j} - \vec{v}_{i} \right)
	\label{eq:kij}
\end{equation}

\noindent Z~definice \ref{eq:kij} je jasné, že platí vztahy $\vec{R}_{ji} = -\vec{R}_{ij}$ a $\vec{R}_{ii} = 0$. Vzorec pro výpočet $K_{ji}$ se obecně liší podle toho jaké typy fází spolu interagují. Pokud se jedná o~interakci kapalina-pevná fáze, tak koeficient mezifázového sdílení hybnosti má tvar:

\begin{equation}
	K_{fs}= \frac{3\alpha_{s}C_{D}\rho_{f}|\vec{v}_{s} - \vec{v}_{f}|}{4d_{s}}
	\label{eq:kfs}
\end{equation}
  
\noindent Symbol $d_{s}$ značí průměr pevných částic a $C_{D}$ koeficient odpor, kterým se právě ovlivňuje chování odporové síly. Tento koeficient může být obecně složitou funkcí relativní rychlosti pohybu částice, její velikosti, hustotou a viskozitou tekutiny případně dalších veličin.   

V~současnosti existuje řada modelů pro koeficient odporů, které se liší podle toho, pro jaký účel byly navrženy. Jedním z~nejznámějších je model, který navrhli \citet{schi32}. Autoři získali vztah \ref{eq:schlneu} pro výpočet koeficientu odporu na základě experimentálního měření rychlosti usazování sférické částice ve stagnantním sloupci kapaliny.    
 
	\hypertarget{hyp:schlneu}{} 
\begin{equation}
	\label{eq:schlneu}
  C_{D0} = \left\{ \begin{array}{ll}
  \frac{24}{Re_{p}}  \left( 1 + \num{0.15}Re_{p}^{\num{0.687}} \right) & Re_{p} \le 1000\\
  \num{0.44} & Re_{p} > 1000\\
  \end{array} \right.
\end{equation}

\noindent Ze vztahu \ref{eq:schlneu} je patrné, že Schillerova-Naumannova korelace je funkcí Reynoldsova kritéria pro částici, které je definováno jako:

\begin{equation}
	Re_{p}= \frac{\rho_{f}d_{s}|\vec{v}_{s} - \vec{v}_{f}|}{\eta_{f}}
	\label{eq:reyp}
\end{equation}

Bohužel odporový koeficient navržený Schillerem a Naumannem se ukázal jako nepříliš vhodný k~popisu odporové síly v~systémech s~plně vyvinutým turbulentním prouděním. Proto řada autorů navrhl úpravu této korelace a některé z~nich jsou uvedeny v~tab. \ref{tab:cds}. Člen $\lambda$ se nazývá Kolmogorovo mikroměřítko je funkcí kinematické viskozity a disipace energie a určuje nejmenší velikost turbulentní vírů přítomných v~systému. 

\begin{table}[h!]
\begin{center}
  
		\hypertarget{hyp:cds}{}
		\caption{Modely pro odporový koeficient v~turbulentní oblasti proudění}
		\label{tab:cds}
\begin{tabular}{|c|c|>{\centering\arraybackslash}p{5cm}|}
  \hline
  
{\textbf{Autor}} & {\textbf{Koeficient odporu}} & {\textbf{Stanoveno na základě}} \\ \hline{}

\citet{pin01} & $C_{D} = C_{D0} \left( \num{0.4}\tanh\left(  \frac{16\lambda}{d_{s}} - 1  \right) \right) ^{-2}$ & měření rychlosti usazování v~míchací nádobě \\ \hline
 
\citet{bru98} & $C_{D} = C_{D0} \left( 1 + \num{8.76e-4} \left( \frac{\lambda}{d_{s}} \right)^{3} \right)$ & experimentální studie Taylorova–Couetteova toku \\ \hline 

\citet{kho06} & $C_{D} = C_{D0} \left( 1 + \num{8.76e-5} \left( \frac{\lambda}{d_{s}} \right)^{3} \right)$ & úpravy Brucatova vztahu pro míchací nádoby  \\ \hline 

\end{tabular}
\end{center}
\end{table}

\vspace{-9mm}








